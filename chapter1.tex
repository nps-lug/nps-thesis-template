\chapter{[Chapter one Title]}\label{ch:chapter1}

This is the beginning of your thesis. Always have text between every head 
and subhead. Here are some citations\footnote{These citations were taken
from the IEEEtran example bib file.} that demonstrate nearly every bib 
style\footnote{\lipsum[10]} that exists~\cite{IEEEhowto:IEEEtranpage,IEEEexample:shellCTANpage,
IEEEexample:IEEEwebsite,IEEEexample:bibtexuser,IEEEexample:bibtexdesign,
IEEEexample:tamethebeast,IEEEexample:bibtexguide,
IEEEexample:article_typical,IEEEexample:articleetal,IEEEexample:conf_typical,
IEEEexample:book_typical,IEEEexample:articlelargepages,
IEEEexample:articledualmonths,IEEEexample:TBPmisc,IEEEexample:TBParticle,
IEEEexample:bookwitheditor,IEEEexample:book,IEEEexample:bookwithseriesvolume,
IEEEexample:inbook,IEEEexample:inbookpagesnote,IEEEexample:incollection,
IEEEexample:incollectionwithseries,IEEEexample:incollection_chpp,
IEEEexample:incollectionmanyauthors,
IEEEexample:motmanualhowpub,IEEEexample:confwithadddays,
IEEEexample:confwithvolume,IEEEexample:confwithpaper,
IEEEexample:confwithpapertype,IEEEexample:presentedatconf,
IEEEexample:masters,IEEEexample:masterstype,IEEEexample:phdurl,
IEEEexample:techrep,IEEEexample:techreptype,IEEEexample:techreptypeii,
IEEEexample:techrepstdsub,IEEEexample:unpublished,IEEEexample:electronhowinfo,
IEEEexample:electronhowinfo2,IEEEexample:electronorgadd,IEEEexample:uspat,
IEEEexample:jppat,IEEEexample:frenchpatreq,IEEEexample:periodical,
IEEEexample:standard,IEEEexample:standardproposed,IEEEexample:draftasmisc,
IEEEexample:miscforum,IEEEexample:whitepaper,IEEEexample:datasheet,
IEEEexample:private,IEEEexample:miscrfc,IEEEexample:softmanual,
IEEEexample:softonline,IEEEexample:miscgermanreg,IEEEexample:bluebookstandard
}.

\section{A Section}\label{sec:something}
\lipsum[1-4] % remove me

\begin{figure}
\fbox{\parbox{\textwidth}{\lipsum[1]}} % example figure (text in a box)
\caption{[Figure Name]}
\end{figure}

\section{Another Section}
\lipsum[2-3] % remove me

\subsection{A subsection}
\lipsum[5-6] % remove me

\section{Yet Another Section}
\lipsum[1] % remove me
\begin{table}
\begin{center}
\begin{tabular}{ l c r }
  1 & 2 & 3 \\
  4 & 5 & 6 \\
  7 & 8 & 9 \\
\end{tabular}
\end{center}
\caption{[Table Name]}
\end{table}
\lipsum[2]

\subsection{A subsection}\label{sec:another}
\lipsum[3]

\subsubsection{A subsubsection}\label{sec:minorstuff}
\lipsum[4]


\section{Things to Remember When Writing}\label{sec:remember}
\begin{enumerate}
\item Punctuation (periods and commas) go inside quotation marks. 
\item Use the \LaTeX{} \verb+\begin{figure}+ and \verb+\begin{table}+ environment to
  create floating figures and tables. Use the \verb+\caption+ command
  to create your captions. Label your captions with the
  \verb+\label{foo}+ command inside the caption itself. Reference
  these figures and tables with the \verb+\ref{foo}+ reference command.
\item When using \emph{i.e.,} \emph{e.g.,} or \emph{etc.}, always put
  a comma before and after, \emph{e.g.}, like this.
\item Do not split text around a figure or table. 
\item Master's degree has an apostrophe and Postgraduate is one word. 
\item If you use ``however,'' make sure there's a comma before and after,
unless you 
start a sentence with it. However, it's best not to start a sentence
with ``however.'' And while we're on the subject, you should try to avoid starting a sentence with ``and'' or ``because.'' 
\item When typing a date, do not use ``st'' or ``th.'' Instead, just
  note the date: July 4, 1776, is Independence Day. Commas go 
after Month/date, year: both Jefferson and Adams died on July 4, 1826.
No comma between month/yr: \textit{Alice's Adventures in Wonderland} was published in July 1865.
\item Spell out numbers 1 through 9. 
\item Use automatic numbering and lettering.
\item Capitalize C in Chapter, F in Figure and T in Table when referring
to chapters, 
figures or tables in the text. Better yet, use the \verb+\chapref+,
\verb+\figref+ and \verb+\tabref+ commands in the NPS report template.
\item When there is more than one reference, put them both into the \verb+\cite+ command: \verb+\cite{john1,john2}+. It will render like this \cite{IEEEhowto:IEEEtranpage,IEEEexample:shellCTANpage}.
\item Avoid writing in the first person!
\item Make sure there's no widows at the
  top of the page---if \LaTeX{} gives you a hard time, you may need to
  add or remove text so that everything works out properly.
\item Footnote numbers go outside the punctuation. 
\item When typing equations in text and use ``where'' or ``if.'' Use
  Math Mode. 
\item When inserting symbols, use Math Mode.
\end{enumerate}
