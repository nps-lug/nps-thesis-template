\chapter{Helpful Writing Tips for Your Report or Thesis}
This chapter discusses elements of writing and style that are helpful
when writing a report or thesis at NPS. This chapter is based on a
publication that was distributed by the NPS Thesis Processor in 2009.

\section{English Grammar Tips}
\begin{enumerate}
\item Punctuation (periods and commas) go inside quotation marks. 
\item When using \ie \eg or \etc always put
  a comma before and after, \emph{e.g.}, like this. You can also use
  the |\ie|, |\eg| and |\etc| macros that the thesis template provides.
\item Master's degree has an apostrophe and Postgraduate is one word. 
\item If you use ``however,'' make sure there's a comma before and after,
  unless you start a sentence with it. However, it's best not to start a sentence
  with ``however.'' And while we are on the subject, you should try to avoid 
  starting a sentence with ``and'' or ``because.'' 
\item When typing a date, do not use ``st'' or ``th.'' Instead, just
  note the date: July 4, 1776, is Independence Day. Commas go 
  after Month/date, year. No comma between month/yr. 
\item Spell out numbers 1 through 9 as one to nine.  Larger numbers remain as digits.  
\item Capitalize C in Chapter, F in Figure and T in Table when
  referring to chapters, figures or tables in the text and use roman
  numerals vs numbers or spelling out, etc. for chapters. Even
  better, use the referencing commands described in \secvref{refcommands}.
\item Footnote numbers go outside the punctuation. 
\item Ibid cannot be the first footnote on the page.
\item When typing equations in text and when using ``where'' or ``if,''
  \etc and it's not a new paragraph the word starts at the margin.
\item When inserting symbols, use the proper symbols commands.  Avoid trying to
  include the character directly in the |.tex| file.
\item Avoid writing in the first person!
\item Avoid dangling participles.  Wrong: Substituting (12) into (14) gives...;
  Correct: Substituting (12) into (14), we get...
\item Contractions are not used in formal writing.  Cannot is one word.
\item Chapters, figures and tables do not show things.  Instead, things are shown
  or illustrated in figures and tables.  Things are discussed in chapters or sections.
\end{enumerate}

\section{Additional Writing Tips}
\begin{enumerate}
\item Displayed equations must be numbered, part of a sentence and properly punctuated.  This means
  your equation may have a period as the last character to indicate the sentence has ended. 
\item In-line equations in paragraphs must be simple and use ``/'' to indicate division.
\item All figure captions should be complete sentences with a period at the end of the caption.
\item Figures and tables should display the units associated with quantities being displayed.
\item Axes in figures should be clearly labeled with quantities and units.
\item Discuss all figures and tables in your thesis.
\item Acronyms need to be defined in the acronyms list, in the abstract, in the executive
  summary, and the first time they are used in the thesis.  They do not need to be
  defined for every chapter.
\item The introduction should provide the background that allows the reader to
  understand why he or she should be interested in the problem.  Provide a
  discussion of related work with references.  State clearly and explicitly the
  goal(s) or objective(s) of your work.  Discuss how your work differs from the
  previous works.
\item Abstracts briefly summarize the work and help the reader to ascertain the purpose
  of the thesis.  An abstract may include the problem at hand, the technique used to solve
  the problem, and indicate the conclusion of the results.
\item Some departments require an an additional section called the Executive Summary.
  It is more comprehensive than the abstract and generally 2-10 pages in length.  The
  Executive Summary must stand alone from the rest of the document.  Figures and tables
  are numbered independently from the thesis content and do not appear in the List of
  Figures or List of Tables.  Additionally, references in the Executive Summary are
  independent from the thesis and there is a separate list of references at the
  end of the Executive Summary.   
\item Conclusions summarize the results obtained in your research and emphasize
  your original contributions.  Recommended future work should include any new
  questions arising from your research.
\end{enumerate}

\section{\LaTeX{} Tips}
\begin{enumerate}
\item Do not use ``*'' or ``x'' to indicate multiplication.  $X=YZ$ is sufficient.
\item If you must use multiplication, please do so using |\times| in
  math mode. That is, type |$X=Y\times Z$| to produce $X=Y\times Z$.
\item Use the \LaTeX{} \verb|\begin{figure}| and \verb|\begin{table}| environment to
  create floating figures and tables. Use the |\caption| command
  to create your captions. Label your captions with the
  |\label{marker}| command inside the caption itself. Captions are shown
  in the paper as text.  Labels are internal \LaTeX{} identifiers that can
  be referenced with the |\ref{marker}| reference command.
\item Do not split text around a figure or table.  Write complete paragraphs,
  since \LaTeX{} will place figures in your document to efficiently use the paper.  
\item When there is more than one reference, put them all into the 
  \verb|\cite| command: \\ \verb|\cite{john1,john2,john3}|.
\item Make sure there's at least one and a half lines of text at the
  top of the page---if \LaTeX{} gives you a hard time, you may need to
  add or remove text so that everything works out properly.
\item Don't use math mode as a general italics---use |\emph{}|. 
\item Do not make tables too wide in columns or they can be drawn off the right-side of the paper.
\item Use automatic numbering and lettering by using the appropriate environments, such as
  \texttt{enumerate}, \texttt{itemize} or \texttt{list}.
\item There can only be one |label| entry for a section, figure, \etc.  Trying to have more than one will
  cause a problem in the automatic numbering.  If you need to troubleshoot numbering, look in the generated |.toc| files.
\end{enumerate}

