\chapter{Introduction to Graphing with \LaTeX}
In this (brief) chapter, we show several approaches for constructing
bar graphs that present scientific information. This chapter is
designed to provide you with a starting point for looking at graphing
packages. We expect that you will use this chapter to learn about the
various options available, and then review the documentation
associated with that option once you have made a decision of how you
wish to proceed.

There are many, many options for presenting numeric information in
graphical form using \LaTeX. Fundamentally, though, all of the options
fall in one of two categories:

\begin{enumerate}
\item You can create the graphics entirely within \LaTeX.
\item You can create the graphics with a second package and include
  the graphics file in \LaTeX{} with an |\includegraphics| command.
\end{enumerate}

In this section, we will assume that you wish to make two graphs.

\begin{description}
\item[Graph \#1---Accumulated Spending] This graph will show the
  amount of money spent on a project from January through May, the
  total amount spent, and the total amount allocated for calendar
  year. The data for this graph is shown in \tabref{spending}.
\item[Graph \#2---Scatterplot of (x,y)] This graph will show ten
  values of (x,y) and a line that fits them.
\end{description}
