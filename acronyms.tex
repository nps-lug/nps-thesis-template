
%% The NPS template readily supports two types of list of acronyms.
%% Option 1 - Basic list of acronyms
%% In your document, you are responsible to write out the long form on the
%% first use, then you can simply use the acronym in your paragraphs that
%% follow.  This can become cumbersome during revisions to your thesis
%% because it is your responsibility to know if the acronym has been used yet.
\begin{description}


% NOTE: Please follow the capitalization from the "DoD Dictionary of Military
% and Associated Terms" (http://www.dtic.mil/doctrine/new_pubs/jp1_02.pdf)
% or from common use.
\item[MASINT] measurement and signature intelligence
\item[NPS] Naval Postgraduate School
\item[SRWBR] short range wide band radio
\item[TCP] Transmission Control Protocol
\item[UDP] User Datagram Protocol
\item[USG] United States Government
\end{description}


%% Option 2 - Powerful list of acronyms
%% The acronym option must be used on the documentclass line in thesis.tex
%% LaTeX tracks the usage of acronyms and uses the long form on their 
%% first use.  This ensures consistency throughout your document even 
%% in different revisions of your document.
%% Your writing is also more portable to a journal or conference paper.
%% 
%% The NPS template will only print items in this list that are actually
%% used in the paper.  In this manner, your acronym list will always be 
%% up to date.  You can then reuse this acronyms.tex in other documents
%% as you continue to expand your acronyms in use.
%%
%% On the first line below, the longest acronym is placed in the 
%% square brackets.  This is used to size the column of the table 
%% correctly.
%\begin{acronym}[XXX] % longest acronym in these square brackets
%\acro{NPS}{Naval Postgraduate School}
%\acro{US}{United States}
%\acro{DoD}{Department of Defense}
%\acro{TCP}{Transmission Control Protocol}	
%\acro{UDP}{User Datagram Protocol}	
%\end{acronym}

%% Additional notes on option 2

%% To use your acronym in a paragraph use \ac{shortname} or \acplural{shortname}
%% \acp{shortname} for plural shorthand.
%% Ex:  The \ac{FFT2} is performed on the data. 
%% The \acplural{FFT2} are efficient.
%% This command is how LaTeX tracks their usage.

%% If you are also using the index option, you may want to use this:
%% \newcommand{\FFT2}{\ac{FFT2}\index{fast Fourier Transform}\xspace}
%% to save some time and typing.  Then you can do:
%% Ex: The \FFT2 is performed on the data.
%% The acronym will be used correctly, and the index will contain the 
%% use of the keyword. 
%%
%% The command's \xspace ensures that the spacing after the word is
%% handled correctly (ie, last word of a sentence gets a period on 
%% the word).


