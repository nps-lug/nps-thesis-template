% This starts the document; the "npsreport" style is really a modified
% report style. Feel free to use the options explained in the technical
% report NPS-CS-11-011, included under the doc/ directory.
\documentclass[twoside,thesis]{npsreport}

%
% Put extra packages you may need to customize your thesis
%
\usepackage{doc,lipsum} % provides \BibTex and \lipsum macros, for demos

%
% For Example: you might find one of these useful:

%\usepackage{epstopdf}        % to use .eps files for figures
%\usepackage{bm}              % bold math if you need bold greek letters
%\usepackage{glossaries}      % see http://en.wikibooks.org/wiki/LaTeX/Glossary
%\usepackage{asymptote}       % for graphics
% The asymptote package allows for very nice graphics and figures
% Proper usage requires additional information located at:
% http://asymptote.sourceforge.net/
% See the gallery at this URL for examples

%\usepackage{placeins}        % float placement
% Provides \FloatBarrier which keeps figures/tables in the same section.
% LaTeX sometimes moves them to fill up pages.
% http://ftp.math.purdue.edu/mirrors/ctan.org/macros/latex/contrib/placeins/placeins-doc.pdf

%\usepackage[numbered]{mcode} % matlab code
% The mcode package must be separately downloaded.
% http://www.mathworks.com/matlabcentral/fileexchange/8015-m-code-latex-package

%\usepackage{flafter}         % float placement
% Ensures that figures/tables do not appear in the document before
% they are referenced in the text.

% This package lets us build references that appear after the Executive Summary
\usepackage{bibunits}


\title{[Title]}

% Student info
\author{[Author Name]}
\rank{[Rank, Service]}    %\rank{Civilian} % if you don't have a rank
\degree{Master of Science in [Degree]}
\degreeabbreviation{MS}   % Should be MS, MBA or MA
\prevdegrees{[B.S., My Old School, Year]} % previous degree

% Department info
\department{Department of [Department]}
\thesisadvisor{[Primary Advisor]}
\secondreader{[Second Reader]}
\departmentchair{[Department Chair]}

% The date you are graduating:
\degreedate{[Month Year]}

% See Thesis processor's release form for approved distribution statements.
\distribution{Approved for public release; distribution is unlimited}

% Your abstract.  New paragraphs start after an empty line.
\abstract{%
\lipsum[1] % example text, remove me
}

% Switch the below lines around, if FOUO
\securitybanner{}  %\securitybanner{FOR OFFICIAL USE ONLY}

%
% Mandatory fields for the SF298.
%
\ReportType{Master's Thesis}
\ReportDate{MM-DD-YYYY}         % for a thesis, graduation date
\SponsoringAgency{N/A}          % really, for technical reports
\DatesCovered{MM-DD-YYYY to MM-DD-YYYY}
\ReportClassification{Unclassified}
\AbstractClassification{Unclassified}
\PageClassification{Unclassified}
%
% Optional fields for the SF298.
%
\RPTpreparedFor{}
\ContractNumber{}
\GrantNumber{}
\ProgramElementNumber{}
\TaskNumber{}
\WorkUnitNumber{}
\POReportNumber{}
\Acronyms{}
\SMReportNumber{}
\SubjectTerms{}
\ResponsiblePerson{}
\RPTelephone{}
\SignatureOne{}
\SignatureTwo{}
\SupplementaryNotes{The views expressed in this document are those of
  the author and do not reflect the official policy or position of the
  Department of Defense or the U.S. Government. %
  IRB Protocol Number: N/A. % if you need to note an IRB Protocol or N/A
}

% Optional. Prevents footnotes from being reset at each chapter
% Comment this out to have them reset with each chapter.
\makeatletter
\@removefromreset{footnote}{chapter}
\makeatother

% Optional. Adds pdf metadata and links.
% This should be right before the \begin{document}, to be the
% last package / macros defined. (Hyper-ref is fragile,
% needs to be last, and has known conflicts with other packages.)
% Comment out if you have build problems building with hyperref
\NPShyperref

%
% Your thesis begins here
%
\begin{document}

\NPScover                  % Cover page
\NPSsftne                  % SF298 form
%\NPSsignature             % Tech Report page (iii): signature page
\NPSthesistitle            % Thesis page (iii): title page
\NPSabstractpage           % Abstract Page
\NPSfrontmatter            % NPS front matter follows

% This changes the chaptermark and includes the various tables
% It must be here.
\renewcommand{\chaptermark}[1]{\markboth{\MakeUppercase{\chaptername}\ \thechapter.\ #1}{}}

%
% If you don't need one of these, comment it out.
%
\NPStableOfContents
\NPSlistOfFigures
\NPSlistOfTables
\NPSlistOfAcronymsFromFile{acronyms}

%
% Put Executive summary here.
% New paragraphs start after an empty line.
%
\NPSexecsummary{
\begin{bibunit}[nps_thesis] % START: bibunit, use nps_thesis.bst for the style

This is an example of how to create an executive summary with its own references 
section using the \texttt{bibunit} package.
The build process needs to change to accomodate this. 
The \texttt{bibunit} package builds separate unit files
\texttt{bu1.aux}, \texttt{bu2.aux}, etc. 
These needs to be run through Bib\TeX{} separately.
In this example, the executive summary if the first (and only) bibunit, 
so we need to do the commands:
\begin{itemize}
	\item[] \texttt{pdflatex report}
	\item[] \texttt{bibtex report}
	\item[] \texttt{bibtex bu1}
	\item[] \texttt{pdflatex report}
	\item[] \texttt{pdflatex report}
\end{itemize}
The \texttt{Makefile} demonstrates how to script this.

\section*{Executive Summary Section}
The references~\cite{IEEEexample:incollectionmanyauthors},
\cite{IEEEexample:articledualmonths}
are in both the summary,
and in the final references; they are numbered separately.
Some references~\cite{IEEEexample:shellCTANpage,IEEEexample:bibtexuser,
IEEEexample:article_typical} only appear in this section,
and are not part of the final bibliography.

Note, you cannot use numbered sections in the executive summary,
since the summary has no number itself.

This sentence demonstrates that acronyms, like \ac{US}, work in the
executive summary; the \ac{US} is the short version. The counter will be re-set
in the main body, where its first use will be long again.

\subsection*{An Exec Summary Subsection}
\lipsum[2-3] % example text; remove me

%
% this makes references appear at the section-level instead of chapter-level
%
\begingroup
 \let\stdthebibliography\thebibliography
 \renewcommand{\thebibliography}{%
 \let\chapter\subsubsection
 \titlespacing*{\subsubsection}{0pt}{5ex plus 2ex}{-1ex plus .2ex}
 \stdthebibliography}
 \raggedright     % don't automatically full justify bibliographic references
 \singlespacing   % reduce extra line breaks between entries
 \small
 \putbib[thesis]  % use thesis.bib and place the bibliography here
\endgroup
%
% This is the end of special macros that tweak the appearance of the references
%
\end{bibunit}    % END: bibunit
}

%
% Put acknowledgements here.  
% New paragraphs start after an empty line.
%
\NPSacknowledgements{%
\lipsum[1-3] % example text; remove me
}

% Start layout for the NPS body
\NPSbody


% CHAPTERS
% You have two options on how to structure your thesis:
% a) A single file. All chapters, sections, etc. go in this file.
%    This can make navigating your thesis a little more difficult.
% b) Use multiple files.  One chapter per file is recommended.
%    This breaks your thesis up into logical units to edit.
%
\chapter{[CHAPTER ONE TITLE]}
[CHAPTER BODY]

This is the beginning of your thesis. Don't be a Micky
Mouse\cite{mm2}: Always have text between every head and subhead.

\section{Your First Section}
[Section One Body]
\section{Your Second Section}
[Section Two Body]
\section{Your Third Section}
[Section Three Body]


Here are a few more examples of citations to test with, including~\cite{ChungEtAl2011} and~\cite{Anisi2004}.
%
% (include other chapters here...)
%


% APPENDICES
% You have two recommended options for your appendix:
% a) A single appendix (with a single TOC entry)
% b) Multiple appendices. Look under the examples directory for a demo of
%   multiple appendices.
%
\NPSappendixTOC{[My Appendix Title]}
\lipsum[1-4] % example text; remove me



% REFERENCES
% List all your BibTeX reference source files (ending in *.bib extension)
%
\NPSbibliography{thesis}


%
% This is the official end of the thesis.
%
\NPSend

% DISTRIBUTION LIST
% The list is automatically properly numbered
% and already populated with the mandatory recipients.
%
\NPSdistribution{Initial Distribution List}
\begin{distributionlist}
\item Defense Technical Information Center\\Ft. Belvoir, Virginia
\item Dudley Knox Library\\Naval Postgraduate School\\Monterey, California
%
%---- Other entries are no longer needed, because of Special Abstract Form
% Marine Corps students are required to show:
%\item Marine Corps Representative\\Naval Postgraduate School\\Monterey, California
%\item Directory, Training and Education, MCCDC, Code C46\\Quantico, Virginia
%\item Marine Corps Tactical System Support Activity (Attn: Operations Officer)\\Camp Pendleton, California
%
% Officer students in the Operations Research Program are also required to show:
%\item Director, Studies and Analysis Division, MCCDC, Code C45\\ Quantico, Virginia
%
% Officer students in the Space Ops/Space Engineering Program or in the Information Warfare/Information Systems and Operations are also required to show:
%\item Head, Information Operations and Space Integration Branch,\\ PLI/PP\&O/HQMC, Washington, DC
\end{distributionlist}


\end{document}

