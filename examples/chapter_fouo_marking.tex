\chapter{Introduction to the FOUO Thesis}

This chapter demonstrates the \verb|\smark| macro and
\verb|smarkenv| environment for security marking.

\smarkthispage

The following pages should get marked:
\begin{itemize}
\item The first page of the first chapter gets marked no matter what;
you can do this manually with the \verb|\smarkthispage| macro.
\item The front cover; this gets marked for you automagically if the
\verb|securitybanner| macro is defined.
\item Any page holding a marked paragraph; these are marked
automagically, as long as each paragraph is wrapped in the
\verb|\begin{smark}[xxx]| \ldots \verb|\end{smark}| environment,
where \verb|xxx| is the paragraph-level marking, \ie \verb|(FOUO)|.
\item Any page holding a marked figure or table; these are
marked automagically, as long as the figure or table content are
wrapped in the \verb|\begin{smarkenv}{xxx}| \ldots \verb|\end{smarkenv}|
environment, where \verb|xxx| is the parenthesized top and bottom
marking, \ie \verb|(FOR OFFICIAL USE ONLY)|.
\end{itemize}

\textbf{Note:} We cannot handle:
\begin{itemize}
\item FOUO titles in Chapters, Sections, Subsections;
\item FOUO captions in Tables or Figures;
\item FOUO content in the References section.
\end{itemize}
These have auto-generated portions (ToC, LoF, LoT, Bibliography),
and the macros are not ``smart'' enough to detect which of the
generated pages have FOUO content, to mark the lower pages
appropriately. If needed, the author will need to mark the lower
pages of the report manually in these cases.

\newpage
\section{A Section with Restricted Content}

\begin{smark}[(FEUO)]
This paragraph is unclassified and has a FEUO restriction.
Only put one paragraph at a time in this environment.
Each paragraph must be marked separately.
\lipsum[2]
\end{smark}

This is a new paragraph and is not marked.
\lipsum[3]

\newpage
\lipsum[1-3]

\begin{smark}[(FEUO)]
This paragraph is unclassified and has a FEUO restriction.
It crosses two pages. Both pages should be marked.
\lipsum[4]
\end{smark}

This is a new paragraph and is not marked.
\lipsum[3]

\newpage
\section{A Chapter with Marked Figures}
\lipsum[1-4]

\begin{figure}
\begin{smarkenv}{(FEUO)}
\framebox[\textwidth]{\parbox{\textwidth}{\lipsum[1]}}
\end{smarkenv}
\caption{[Figure with unrestriced caption, FEUO content]}\label{fig:mark}
\end{figure}

Two special notes on marking floats (figures and tables):
\begin{enumerate}
\item When the float content is restricted, the marking abbreviation is
used (see Figure~\ref{fig:mark}); this is an exception
to the rules for marking classified figures or tables, where the
marking is spelled out;
\item Because floats may ``drift'' to a second
page, the marking macros may sometimes mark the bottom portions
of pages, even if they don't contain marked content. This can
be corrected by adjusting the placement of the float, in most
cases.
\end{enumerate}
